The navicom package intends to provide standard methods to visualise high-\/throughput data in NaviCell web service. It also provide several processing method to get some extra meaning out of the data.

\begin{DoxyAuthor}{Author}
Mathurin Dorel 
\end{DoxyAuthor}
\begin{DoxyParagraph}{License: GNU Public License.}

\end{DoxyParagraph}
\hypertarget{main_start}{}\subsection{Getting started}\label{main_start}
The communication with NaviCell web service, and data processing are performed by the \hyperlink{classnavicom_1_1navicom_1_1NaviCom}{NaviCom} class, which can be initialised with a simple NaviCell map URL. 
\begin{DoxyCode}
nc = NaviCom(map\_url=\textcolor{stringliteral}{'https://navicell.curie.fr/navicell/maps/cellcycle/master/in
      dex.php'})
\end{DoxyCode}
 Data and annotations can then be loaded in the \hyperlink{classnavicom_1_1navicom_1_1NaviCom}{NaviCom} object. 
\begin{DoxyCode}
nc.loadData(\textcolor{stringliteral}{"data/Ovarian\_Serous\_Cystadenocarcinoma\_TCGA\_Nature\_2011.txt"})
\end{DoxyCode}
 Two formats are accepted: \begin{DoxyItemize}
\item a matrix format where data are represented as matrix, with explicit rows and columns names. The first row has to start by GENE (for data) or NAME (for annotations); \item a set of matrix format where each matrix is separated by a header line expliciting the type of data and starting with M (data) or ANNOTATIONS (annotations).\end{DoxyItemize}
After being loaded, data and annotations can be easily controlled: 
\begin{DoxyCode}
nc.listData()
nc.listAnnotations()
\end{DoxyCode}
\hypertarget{main_display_config}{}\subsubsection{Display configuration}\label{main_display_config}
The displays in the NaviCell map can be easily configured with the \hyperlink{classnavicom_1_1displayConfig_1_1DisplayConfig}{DisplayConfig} class. It can be provided to the \hyperlink{classnavicom_1_1navicom_1_1NaviCom}{NaviCom} class. 
\begin{DoxyCode}
display\_config = DisplayConfig(5, zero\_color=\textcolor{stringliteral}{"000000"}, na\_color=\textcolor{stringliteral}{"ffffff"}, na\_size
      =0)
nc = NaviCom(map\_url=\textcolor{stringliteral}{'https://navicell.curie.fr/navicell/maps/cellcycle/master/in
      dex.php'}, display\_config)
\end{DoxyCode}
\hypertarget{main_extra_data}{}\subsubsection{Adding extra data}\label{main_extra_data}
Data in the \hyperlink{classnavicom_1_1navicom_1_1NaviCom}{NaviCom} class are represented in the \hyperlink{classnavicom_1_1navidata_1_1NaviData}{NaviData} format, which is a matrix of data that can be indexed by row and column names. \hyperlink{classnavicom_1_1navidata_1_1NaviData}{NaviData} objects can be created independently and integrated to a \hyperlink{classnavicom_1_1navicom_1_1NaviCom}{NaviCom} object: 
\begin{DoxyCode}
extra\_data = NaviData(data\_matrix, row\_names, col\_names, method, processing)
nc.bindNaviData(extra\_data, \textcolor{stringliteral}{"extra\_data"})
\end{DoxyCode}
 Each datatable is identified by its method, which is the biological or computationnal method used to generate the data, and its processing, which is a NaviCell map related processing to tweak the visualisation of the data. The default processing for unprocessed data is 'raw".\hypertarget{main_visualisation}{}\subsection{Data visualisation}\label{main_visualisation}
The \hyperlink{classnavicom_1_1navicom_1_1NaviCom}{NaviCom} class provides several method to display the data in NaviCell: \begin{DoxyItemize}
\item {\bfseries display} Generic display function to perform any kind of personnalised display \item {\bfseries displayOmics} Display -\/omics data as map staining with extra information or data on using the other display modes \end{DoxyItemize}
